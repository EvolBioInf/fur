The design of diagnostic PCR primers is often hampered by an excess of
candidates that also amplify off-target regions. To minimize the
chance of cross-amplification, primers should be designed from
template sequences that are unique to the target strain. The program
\texttt{fur} \emph{finds unique regions} by comparing the genomes of a
sample of target strains to the genomes of the closest relatives the
targets are to be distinguished from. The underlying heuristic is that
any region that distinguishes a target from its closest relatives,
also distinguishes it from all other sequences out there.

Consider, for example, \textit{Escherichia coli} ST131, a multi-drug
resistant strain that causes urinary tract and blood infections in
humans~\cite{pet14:glo}. \emph{E. coli} ST131 belongs to the B2
phylogenetic subgroup, which corresponds to serotype O25b:H4.
Figure~\ref{fig:eco} shows the phylogeny of 105 \emph{E. coli} B2
strains. The clade marked ST131 comprises 95 strains newly sequenced
by~\cite{pet14:glo}, plus three STS131 reference genomes, SE15, NA114,
and EC958. This clade defines the \emph{targets} marked $\mathcal{T}$
in Figure~\ref{fig:eco}. The seven remaining
\emph{E. coli} strains are the \emph{neighbors}, $\mathcal{N}$. They also
belong to the B2 group, but not to ST131~\cite{pet14:glo}. The aim is
to find regions specific to ST131. In Section~\ref{sec:furTut} a
tutorial-style analysis of this data set shows how to do this
using \texttt{fur}.

    \begin{figure}
      \begin{center}
	\tiny
	\resizebox{\textwidth}{!}{\begin{pspicture}(0.000,0.000)(11.358,15.500)
\psline(9.699320,15.400000)(11.358173,15.400000)
\rput(10.528746,15.600000){\normalsize 0.001}
\rput(11.358172,0.000000){\rnode{n1}{}}\uput{4pt}[0.000000]{0.000000}(n1){E2348 69}
\rput(9.511868,0.144231){\rnode{n3}{}}\uput{4pt}[0.000000]{0.000000}(n3){ED1a}
\rput(9.138626,0.288462){\rnode{n5}{}}\uput{4pt}[0.000000]{0.000000}(n5){S88}
\rput(9.045730,0.432692){\rnode{n7}{}}\uput{4pt}[0.000000]{0.000000}(n7){APEC O1}
\rput(8.750454,0.576923){\rnode{n9}{}}\uput{4pt}[0.000000]{0.000000}(n9){UTI89}
\rput(7.912733,0.504808){\rnode{n8}{}}
\rput(7.579303,0.396635){\rnode{n6}{}}
\rput(1.728526,0.270433){\rnode{n4}{}}
\rput(8.893115,0.721154){\rnode{n11}{}}\uput{4pt}[0.000000]{0.000000}(n11){536}
\rput(8.194739,0.865385){\rnode{n13}{}}\uput{4pt}[0.000000]{0.000000}(n13){CFT073}
\rput(1.350307,0.793269){\rnode{n12}{}}
\rput(1.209304,0.531851){\rnode{n10}{}}
\rput(10.585147,1.009615){\rnode{n15}{}}\uput{4pt}[0.000000]{0.000000}(n15){S5EC}
\rput(10.616666,1.153846){\rnode{n17}{}}\uput{4pt}[0.000000]{0.000000}(n17){S94EC}
\rput(10.392720,1.081731){\rnode{n16}{}}
\rput(10.381107,1.298077){\rnode{n19}{}}\uput{4pt}[0.000000]{0.000000}(n19){S120EC}
\rput(10.415943,1.442308){\rnode{n21}{}}\uput{4pt}[0.000000]{0.000000}(n21){S31EC}
\rput(10.790844,1.586539){\rnode{n23}{}}\uput{4pt}[0.000000]{0.000000}(n23){S34EC}
\rput(10.372812,1.514423){\rnode{n22}{}}
\rput(10.165457,1.406250){\rnode{n20}{}}
\rput(10.012842,1.243990){\rnode{n18}{}}
\rput(9.963078,1.730769){\rnode{n25}{}}\uput{4pt}[0.000000]{0.000000}(n25){SE15}
\rput(10.480639,1.875000){\rnode{n27}{}}\uput{4pt}[0.000000]{0.000000}(n27){S37EC}
\rput(10.216882,2.019231){\rnode{n29}{}}\uput{4pt}[0.000000]{0.000000}(n29){S26EC}
\rput(10.384426,2.163462){\rnode{n31}{}}\uput{4pt}[0.000000]{0.000000}(n31){S2EC}
\rput(9.958101,2.091346){\rnode{n30}{}}
\rput(9.873499,1.983173){\rnode{n28}{}}
\rput(9.712590,1.856971){\rnode{n26}{}}
\rput(9.548364,1.550481){\rnode{n24}{}}
\rput(10.228494,2.307693){\rnode{n33}{}}\uput{4pt}[0.000000]{0.000000}(n33){S21EC}
\rput(9.963078,2.451923){\rnode{n35}{}}\uput{4pt}[0.000000]{0.000000}(n35){S105EC}
\rput(10.046020,2.596154){\rnode{n37}{}}\uput{4pt}[0.000000]{0.000000}(n37){S22EC}
\rput(10.165457,2.740385){\rnode{n39}{}}\uput{4pt}[0.000000]{0.000000}(n39){S114EC}
\rput(10.268306,2.884615){\rnode{n41}{}}\uput{4pt}[0.000000]{0.000000}(n41){HVM1147}
\rput(10.258353,3.028846){\rnode{n43}{}}\uput{4pt}[0.000000]{0.000000}(n43){S24EC}
\rput(10.241764,2.956731){\rnode{n42}{}}
\rput(10.215222,3.173077){\rnode{n45}{}}\uput{4pt}[0.000000]{0.000000}(n45){S104EC}
\rput(10.405991,3.317308){\rnode{n47}{}}\uput{4pt}[0.000000]{0.000000}(n47){S32EC}
\rput(10.215222,3.245193){\rnode{n46}{}}
\rput(10.080855,3.100962){\rnode{n44}{}}
\rput(10.004548,2.920673){\rnode{n40}{}}
\rput(9.956442,2.758414){\rnode{n38}{}}
\rput(9.913311,2.605168){\rnode{n36}{}}
\rput(10.447462,3.461539){\rnode{n49}{}}\uput{4pt}[0.000000]{0.000000}(n49){S19EC}
\rput(10.255035,3.605770){\rnode{n51}{}}\uput{4pt}[0.000000]{0.000000}(n51){S6EC}
\rput(9.906675,3.533655){\rnode{n50}{}}
\rput(9.707613,3.069411){\rnode{n48}{}}
\rput(9.704295,2.688552){\rnode{n34}{}}
\rput(10.563582,3.750001){\rnode{n53}{}}\uput{4pt}[0.000000]{0.000000}(n53){S128EC}
\rput(11.044649,3.894232){\rnode{n55}{}}\uput{4pt}[0.000000]{0.000000}(n55){S79EC}
\rput(10.021137,3.822116){\rnode{n54}{}}
\rput(10.286553,4.038463){\rnode{n57}{}}\uput{4pt}[0.000000]{0.000000}(n57){HVM277}
\rput(10.401014,4.182694){\rnode{n59}{}}\uput{4pt}[0.000000]{0.000000}(n59){HVM52}
\rput(10.472345,4.326925){\rnode{n61}{}}\uput{4pt}[0.000000]{0.000000}(n61){HVM2044}
\rput(10.472345,4.471156){\rnode{n63}{}}\uput{4pt}[0.000000]{0.000000}(n63){HVM2289}
\rput(10.460733,4.399040){\rnode{n62}{}}
\rput(9.777286,4.290867){\rnode{n60}{}}
\rput(9.727519,4.164665){\rnode{n58}{}}
\rput(9.175121,3.993391){\rnode{n56}{}}
\rput(8.893116,3.340971){\rnode{n52}{}}
\rput(10.528747,4.615386){\rnode{n65}{}}\uput{4pt}[0.000000]{0.000000}(n65){S124EC}
\rput(10.512157,4.759618){\rnode{n67}{}}\uput{4pt}[0.000000]{0.000000}(n67){S129EC}
\rput(10.540359,4.903849){\rnode{n69}{}}\uput{4pt}[0.000000]{0.000000}(n69){S133EC}
\rput(10.474005,4.831733){\rnode{n68}{}}
\rput(10.435850,4.723560){\rnode{n66}{}}
\rput(10.313095,5.048079){\rnode{n71}{}}\uput{4pt}[0.000000]{0.000000}(n71){EC958}
\rput(10.517135,5.192310){\rnode{n73}{}}\uput{4pt}[0.000000]{0.000000}(n73){B36EC}
\rput(10.606713,5.336541){\rnode{n75}{}}\uput{4pt}[0.000000]{0.000000}(n75){HVM1997}
\rput(10.575194,5.480772){\rnode{n77}{}}\uput{4pt}[0.000000]{0.000000}(n77){HVM1619}
\rput(10.643207,5.625003){\rnode{n79}{}}\uput{4pt}[0.000000]{0.000000}(n79){S30EC}
\rput(10.706244,5.769234){\rnode{n81}{}}\uput{4pt}[0.000000]{0.000000}(n81){S134EC}
\rput(10.661454,5.913465){\rnode{n83}{}}\uput{4pt}[0.000000]{0.000000}(n83){S47EC}
\rput(10.663113,6.057696){\rnode{n85}{}}\uput{4pt}[0.000000]{0.000000}(n85){S53EC}
\rput(10.653160,5.985580){\rnode{n84}{}}
\rput(10.644866,5.877408){\rnode{n82}{}}
\rput(10.631596,5.751205){\rnode{n80}{}}
\rput(10.575194,5.615989){\rnode{n78}{}}
\rput(10.546994,5.476265){\rnode{n76}{}}
\rput(10.497229,5.334288){\rnode{n74}{}}
\rput(10.966683,6.201927){\rnode{n87}{}}\uput{4pt}[0.000000]{0.000000}(n87){HVM834}
\rput(10.769280,6.346158){\rnode{n89}{}}\uput{4pt}[0.000000]{0.000000}(n89){S43EC}
\rput(10.556947,6.274042){\rnode{n88}{}}
\rput(10.540359,6.490389){\rnode{n91}{}}\uput{4pt}[0.000000]{0.000000}(n91){S10EC}
\rput(10.551970,6.634620){\rnode{n93}{}}\uput{4pt}[0.000000]{0.000000}(n93){S12EC}
\rput(10.537040,6.562504){\rnode{n92}{}}
\rput(10.495568,6.418273){\rnode{n90}{}}
\rput(10.450780,5.876280){\rnode{n86}{}}
\rput(10.523770,6.778851){\rnode{n95}{}}\uput{4pt}[0.000000]{0.000000}(n95){S101EC}
\rput(10.678044,6.923082){\rnode{n97}{}}\uput{4pt}[0.000000]{0.000000}(n97){S39EC}
\rput(10.445804,6.850966){\rnode{n96}{}}
\rput(10.377791,6.363623){\rnode{n94}{}}
\rput(10.313095,5.705851){\rnode{n72}{}}
\rput(10.187022,5.214705){\rnode{n70}{}}
\rput(10.764303,7.067313){\rnode{n99}{}}\uput{4pt}[0.000000]{0.000000}(n99){IR68}
\rput(10.770938,7.211544){\rnode{n101}{}}\uput{4pt}[0.000000]{0.000000}(n101){IR49}
\rput(10.790845,7.355775){\rnode{n103}{}}\uput{4pt}[0.000000]{0.000000}(n103){IR65}
\rput(10.767621,7.283659){\rnode{n102}{}}
\rput(10.757668,7.175486){\rnode{n100}{}}
\rput(10.513817,7.500006){\rnode{n105}{}}\uput{4pt}[0.000000]{0.000000}(n105){S111EC}
\rput(10.576853,7.644237){\rnode{n107}{}}\uput{4pt}[0.000000]{0.000000}(n107){S130EC}
\rput(10.848905,7.788467){\rnode{n109}{}}\uput{4pt}[0.000000]{0.000000}(n109){S96EC}
\rput(10.532063,7.716352){\rnode{n108}{}}
\rput(10.374473,7.608179){\rnode{n106}{}}
\rput(10.328025,7.391832){\rnode{n104}{}}
\rput(10.399355,7.932698){\rnode{n111}{}}\uput{4pt}[0.000000]{0.000000}(n111){S11EC}
\rput(10.669748,8.076929){\rnode{n113}{}}\uput{4pt}[0.000000]{0.000000}(n113){S1EC}
\rput(10.550311,8.221160){\rnode{n115}{}}\uput{4pt}[0.000000]{0.000000}(n115){S93EC}
\rput(10.440826,8.149045){\rnode{n114}{}}
\rput(10.399355,8.040872){\rnode{n112}{}}
\rput(10.619984,8.365392){\rnode{n117}{}}\uput{4pt}[0.000000]{0.000000}(n117){S109EC}
\rput(10.561923,8.509623){\rnode{n119}{}}\uput{4pt}[0.000000]{0.000000}(n119){S121EC}
\rput(10.598417,8.653853){\rnode{n121}{}}\uput{4pt}[0.000000]{0.000000}(n121){S123EC}
\rput(10.760984,8.798084){\rnode{n123}{}}\uput{4pt}[0.000000]{0.000000}(n123){S113EC}
\rput(10.668089,8.942315){\rnode{n125}{}}\uput{4pt}[0.000000]{0.000000}(n125){S15EC}
\rput(10.556946,9.086546){\rnode{n127}{}}\uput{4pt}[0.000000]{0.000000}(n127){S126EC}
\rput(10.586805,9.230777){\rnode{n129}{}}\uput{4pt}[0.000000]{0.000000}(n129){S132EC}
\rput(10.556946,9.158661){\rnode{n128}{}}
\rput(10.517133,9.050488){\rnode{n126}{}}
\rput(10.508840,8.924286){\rnode{n124}{}}
\rput(10.483957,8.789070){\rnode{n122}{}}
\rput(10.482298,8.649346){\rnode{n120}{}}
\rput(10.367837,8.507369){\rnode{n118}{}}
\rput(10.538700,9.375008){\rnode{n131}{}}\uput{4pt}[0.000000]{0.000000}(n131){S127EC}
\rput(11.192288,9.519238){\rnode{n133}{}}\uput{4pt}[0.000000]{0.000000}(n133){S77EC}
\rput(10.777575,9.663470){\rnode{n135}{}}\uput{4pt}[0.000000]{0.000000}(n135){S99EC}
\rput(10.474005,9.591354){\rnode{n134}{}}
\rput(10.351250,9.483181){\rnode{n132}{}}
\rput(10.301484,8.995275){\rnode{n130}{}}
\rput(10.271624,8.518074){\rnode{n116}{}}
\rput(10.578511,9.807701){\rnode{n137}{}}\uput{4pt}[0.000000]{0.000000}(n137){S131EC}
\rput(11.358173,9.951932){\rnode{n139}{}}\uput{4pt}[0.000000]{0.000000}(n139){S115EC}
\rput(10.885400,10.096163){\rnode{n141}{}}\uput{4pt}[0.000000]{0.000000}(n141){IR18E}
\rput(10.847246,10.240394){\rnode{n143}{}}\uput{4pt}[0.000000]{0.000000}(n143){HVM1299}
\rput(10.837293,10.384624){\rnode{n145}{}}\uput{4pt}[0.000000]{0.000000}(n145){HVM3017}
\rput(10.815728,10.312509){\rnode{n144}{}}
\rput(10.797481,10.528855){\rnode{n147}{}}\uput{4pt}[0.000000]{0.000000}(n147){S118EC}
\rput(10.832316,10.673086){\rnode{n149}{}}\uput{4pt}[0.000000]{0.000000}(n149){S119EC}
\rput(10.797481,10.600970){\rnode{n148}{}}
\rput(10.731127,10.456739){\rnode{n146}{}}
\rput(10.628278,10.276451){\rnode{n142}{}}
\rput(10.405991,10.114191){\rnode{n140}{}}
\rput(10.689654,10.817317){\rnode{n151}{}}\uput{4pt}[0.000000]{0.000000}(n151){S125EC}
\rput(10.812409,10.961548){\rnode{n153}{}}\uput{4pt}[0.000000]{0.000000}(n153){MS2481}
\rput(10.636571,11.105780){\rnode{n155}{}}\uput{4pt}[0.000000]{0.000000}(n155){S122EC}
\rput(10.412626,11.033664){\rnode{n154}{}}
\rput(10.333001,10.925490){\rnode{n152}{}}
\rput(10.263330,10.519841){\rnode{n150}{}}
\rput(10.187022,10.163771){\rnode{n138}{}}
\rput(10.165457,9.340922){\rnode{n136}{}}
\rput(10.155504,8.366377){\rnode{n110}{}}
\rput(10.278259,11.250010){\rnode{n157}{}}\uput{4pt}[0.000000]{0.000000}(n157){HVM5}
\rput(10.273282,11.394241){\rnode{n159}{}}\uput{4pt}[0.000000]{0.000000}(n159){S116EC}
\rput(11.018108,11.538472){\rnode{n161}{}}\uput{4pt}[0.000000]{0.000000}(n161){S103EC}
\rput(10.883740,11.682703){\rnode{n163}{}}\uput{4pt}[0.000000]{0.000000}(n163){S65EC}
\rput(10.387743,11.610588){\rnode{n162}{}}
\rput(10.266646,11.502415){\rnode{n160}{}}
\rput(10.389402,11.826934){\rnode{n165}{}}\uput{4pt}[0.000000]{0.000000}(n165){HVR2496}
\rput(10.412626,11.971165){\rnode{n167}{}}\uput{4pt}[0.000000]{0.000000}(n167){HVM3189}
\rput(10.611690,12.115396){\rnode{n169}{}}\uput{4pt}[0.000000]{0.000000}(n169){P50EC}
\rput(10.351249,12.043280){\rnode{n168}{}}
\rput(10.326366,11.935107){\rnode{n166}{}}
\rput(10.462392,12.259625){\rnode{n171}{}}\uput{4pt}[0.000000]{0.000000}(n171){HVR83}
\rput(10.475662,12.403855){\rnode{n173}{}}\uput{4pt}[0.000000]{0.000000}(n173){P56EC}
\rput(10.371156,12.548086){\rnode{n175}{}}\uput{4pt}[0.000000]{0.000000}(n175){S95EC}
\rput(10.331343,12.475970){\rnode{n174}{}}
\rput(10.404331,12.692316){\rnode{n177}{}}\uput{4pt}[0.000000]{0.000000}(n177){S107EC}
\rput(10.545335,12.836546){\rnode{n179}{}}\uput{4pt}[0.000000]{0.000000}(n179){S135EC}
\rput(10.349589,12.764431){\rnode{n178}{}}
\rput(10.419261,12.980777){\rnode{n181}{}}\uput{4pt}[0.000000]{0.000000}(n181){S108EC}
\rput(10.425897,13.125007){\rnode{n183}{}}\uput{4pt}[0.000000]{0.000000}(n183){MS2493}
\rput(10.724490,13.269237){\rnode{n185}{}}\uput{4pt}[0.000000]{0.000000}(n185){P53EC}
\rput(10.993224,13.413467){\rnode{n187}{}}\uput{4pt}[0.000000]{0.000000}(n187){HVM826}
\rput(10.862175,13.557697){\rnode{n189}{}}\uput{4pt}[0.000000]{0.000000}(n189){S117EC}
\rput(10.628277,13.485582){\rnode{n188}{}}
\rput(10.732785,13.701927){\rnode{n191}{}}\uput{4pt}[0.000000]{0.000000}(n191){NA144}
\rput(11.051284,13.846158){\rnode{n193}{}}\uput{4pt}[0.000000]{0.000000}(n193){P146EC}
\rput(10.822362,13.990388){\rnode{n195}{}}\uput{4pt}[0.000000]{0.000000}(n195){P189EC}
\rput(10.615006,13.918273){\rnode{n194}{}}
\rput(10.542016,13.810101){\rnode{n192}{}}
\rput(10.517134,13.647841){\rnode{n190}{}}
\rput(10.493910,13.458538){\rnode{n186}{}}
\rput(10.425897,13.291772){\rnode{n184}{}}
\rput(10.339637,13.136273){\rnode{n182}{}}
\rput(10.329683,12.950353){\rnode{n180}{}}
\rput(10.545335,14.134619){\rnode{n197}{}}\uput{4pt}[0.000000]{0.000000}(n197){S100EC}
\rput(10.474004,14.278849){\rnode{n199}{}}\uput{4pt}[0.000000]{0.000000}(n199){S110EC}
\rput(10.590123,14.423079){\rnode{n201}{}}\uput{4pt}[0.000000]{0.000000}(n201){S102EC}
\rput(10.505522,14.567309){\rnode{n203}{}}\uput{4pt}[0.000000]{0.000000}(n203){S112EC}
\rput(10.478981,14.495193){\rnode{n202}{}}
\rput(10.459074,14.387021){\rnode{n200}{}}
\rput(10.366179,14.260819){\rnode{n198}{}}
\rput(10.394378,14.711539){\rnode{n205}{}}\uput{4pt}[0.000000]{0.000000}(n205){S92EC}
\rput(10.724490,14.855769){\rnode{n207}{}}\uput{4pt}[0.000000]{0.000000}(n207){S97EC}
\rput(10.659795,15.000000){\rnode{n209}{}}\uput{4pt}[0.000000]{0.000000}(n209){S98EC}
\rput(10.488933,14.927884){\rnode{n208}{}}
\rput(10.341295,14.819712){\rnode{n206}{}}
\rput(10.319730,14.540265){\rnode{n204}{}}
\rput(10.313094,13.745309){\rnode{n196}{}}
\rput(10.291530,13.110641){\rnode{n176}{}}
\rput(10.276600,12.685134){\rnode{n172}{}}
\rput(10.268306,12.310122){\rnode{n170}{}}
\rput(10.238447,11.906268){\rnode{n164}{}}
\rput(10.162140,11.578139){\rnode{n158}{}}
\rput(10.110715,9.972259){\rnode{n156}{}}
\rput(10.055973,7.593482){\rnode{n98}{}}
\rput(8.629358,5.467227){\rnode{n64}{}}
\rput(7.735236,3.508854){\rnode{n32}{}}\uput{4pt}[135.000000]{0.000000}(n32){\normalsize
  ST131}
\rput(0.046448,2.020352){\rnode{n14}{}}
\rput(0.000000,1.010176){\rnode{n2}{}}
\ncangle[angleB=180,angleA=-90,armB=0]{n2}{n1}
\ncangle[angleB=180,angleA=-90,armB=0]{n4}{n3}
\ncangle[angleB=180,angleA=-90,armB=0]{n6}{n5}
\ncangle[angleB=180,angleA=-90,armB=0]{n8}{n7}
\ncangle[angleB=180,angleA=90,armB=0]{n8}{n9}
\ncangle[angleB=180,angleA=90,armB=0]{n6}{n8}
\ncangle[angleB=180,angleA=90,armB=0]{n4}{n6}
\ncangle[angleB=180,angleA=-90,armB=0]{n10}{n4}
\ncangle[angleB=180,angleA=-90,armB=0]{n12}{n11}
\ncangle[angleB=180,angleA=90,armB=0]{n12}{n13}
\ncangle[angleB=180,angleA=90,armB=0]{n10}{n12}
\ncangle[angleB=180,angleA=-90,armB=0]{n14}{n10}
\ncangle[angleB=180,angleA=-90,armB=0]{n16}{n15}
\ncangle[angleB=180,angleA=90,armB=0]{n16}{n17}
\ncangle[angleB=180,angleA=-90,armB=0]{n18}{n16}
\ncangle[angleB=180,angleA=-90,armB=0]{n20}{n19}
\ncangle[angleB=180,angleA=-90,armB=0]{n22}{n21}
\ncangle[angleB=180,angleA=90,armB=0]{n22}{n23}
\ncangle[angleB=180,angleA=90,armB=0]{n20}{n22}
\ncangle[angleB=180,angleA=90,armB=0]{n18}{n20}
\ncangle[angleB=180,angleA=-90,armB=0]{n24}{n18}
\ncangle[angleB=180,angleA=-90,armB=0]{n26}{n25}
\ncangle[angleB=180,angleA=-90,armB=0]{n28}{n27}
\ncangle[angleB=180,angleA=-90,armB=0]{n30}{n29}
\ncangle[angleB=180,angleA=90,armB=0]{n30}{n31}
\ncangle[angleB=180,angleA=90,armB=0]{n28}{n30}
\ncangle[angleB=180,angleA=90,armB=0]{n26}{n28}
\ncangle[angleB=180,angleA=90,armB=0]{n24}{n26}
\ncangle[angleB=180,angleA=-90,armB=0]{n32}{n24}
\ncangle[angleB=180,angleA=-90,armB=0]{n34}{n33}
\ncangle[angleB=180,angleA=-90,armB=0]{n36}{n35}
\ncangle[angleB=180,angleA=-90,armB=0]{n38}{n37}
\ncangle[angleB=180,angleA=-90,armB=0]{n40}{n39}
\ncangle[angleB=180,angleA=-90,armB=0]{n42}{n41}
\ncangle[angleB=180,angleA=90,armB=0]{n42}{n43}
\ncangle[angleB=180,angleA=-90,armB=0]{n44}{n42}
\ncangle[angleB=180,angleA=-90,armB=0]{n46}{n45}
\ncangle[angleB=180,angleA=90,armB=0]{n46}{n47}
\ncangle[angleB=180,angleA=90,armB=0]{n44}{n46}
\ncangle[angleB=180,angleA=90,armB=0]{n40}{n44}
\ncangle[angleB=180,angleA=90,armB=0]{n38}{n40}
\ncangle[angleB=180,angleA=90,armB=0]{n36}{n38}
\ncangle[angleB=180,angleA=-90,armB=0]{n48}{n36}
\ncangle[angleB=180,angleA=-90,armB=0]{n50}{n49}
\ncangle[angleB=180,angleA=90,armB=0]{n50}{n51}
\ncangle[angleB=180,angleA=90,armB=0]{n48}{n50}
\ncangle[angleB=180,angleA=90,armB=0]{n34}{n48}
\ncangle[angleB=180,angleA=-90,armB=0]{n52}{n34}
\ncangle[angleB=180,angleA=-90,armB=0]{n54}{n53}
\ncangle[angleB=180,angleA=90,armB=0]{n54}{n55}
\ncangle[angleB=180,angleA=-90,armB=0]{n56}{n54}
\ncangle[angleB=180,angleA=-90,armB=0]{n58}{n57}
\ncangle[angleB=180,angleA=-90,armB=0]{n60}{n59}
\ncangle[angleB=180,angleA=-90,armB=0]{n62}{n61}
\ncangle[angleB=180,angleA=90,armB=0]{n62}{n63}
\ncangle[angleB=180,angleA=90,armB=0]{n60}{n62}
\ncangle[angleB=180,angleA=90,armB=0]{n58}{n60}
\ncangle[angleB=180,angleA=90,armB=0]{n56}{n58}
\ncangle[angleB=180,angleA=90,armB=0]{n52}{n56}
\ncangle[angleB=180,angleA=-90,armB=0]{n64}{n52}
\ncangle[angleB=180,angleA=-90,armB=0]{n66}{n65}
\ncangle[angleB=180,angleA=-90,armB=0]{n68}{n67}
\ncangle[angleB=180,angleA=90,armB=0]{n68}{n69}
\ncangle[angleB=180,angleA=90,armB=0]{n66}{n68}
\ncangle[angleB=180,angleA=-90,armB=0]{n70}{n66}
\ncangle[angleB=180,angleA=-90,armB=0]{n72}{n71}
\ncangle[angleB=180,angleA=-90,armB=0]{n74}{n73}
\ncangle[angleB=180,angleA=-90,armB=0]{n76}{n75}
\ncangle[angleB=180,angleA=-90,armB=0]{n78}{n77}
\ncangle[angleB=180,angleA=-90,armB=0]{n80}{n79}
\ncangle[angleB=180,angleA=-90,armB=0]{n82}{n81}
\ncangle[angleB=180,angleA=-90,armB=0]{n84}{n83}
\ncangle[angleB=180,angleA=90,armB=0]{n84}{n85}
\ncangle[angleB=180,angleA=90,armB=0]{n82}{n84}
\ncangle[angleB=180,angleA=90,armB=0]{n80}{n82}
\ncangle[angleB=180,angleA=90,armB=0]{n78}{n80}
\ncangle[angleB=180,angleA=90,armB=0]{n76}{n78}
\ncangle[angleB=180,angleA=90,armB=0]{n74}{n76}
\ncangle[angleB=180,angleA=-90,armB=0]{n86}{n74}
\ncangle[angleB=180,angleA=-90,armB=0]{n88}{n87}
\ncangle[angleB=180,angleA=90,armB=0]{n88}{n89}
\ncangle[angleB=180,angleA=-90,armB=0]{n90}{n88}
\ncangle[angleB=180,angleA=-90,armB=0]{n92}{n91}
\ncangle[angleB=180,angleA=90,armB=0]{n92}{n93}
\ncangle[angleB=180,angleA=90,armB=0]{n90}{n92}
\ncangle[angleB=180,angleA=90,armB=0]{n86}{n90}
\ncangle[angleB=180,angleA=-90,armB=0]{n94}{n86}
\ncangle[angleB=180,angleA=-90,armB=0]{n96}{n95}
\ncangle[angleB=180,angleA=90,armB=0]{n96}{n97}
\ncangle[angleB=180,angleA=90,armB=0]{n94}{n96}
\ncangle[angleB=180,angleA=90,armB=0]{n72}{n94}
\ncangle[angleB=180,angleA=90,armB=0]{n70}{n72}
\ncangle[angleB=180,angleA=-90,armB=0]{n98}{n70}
\ncangle[angleB=180,angleA=-90,armB=0]{n100}{n99}
\ncangle[angleB=180,angleA=-90,armB=0]{n102}{n101}
\ncangle[angleB=180,angleA=90,armB=0]{n102}{n103}
\ncangle[angleB=180,angleA=90,armB=0]{n100}{n102}
\ncangle[angleB=180,angleA=-90,armB=0]{n104}{n100}
\ncangle[angleB=180,angleA=-90,armB=0]{n106}{n105}
\ncangle[angleB=180,angleA=-90,armB=0]{n108}{n107}
\ncangle[angleB=180,angleA=90,armB=0]{n108}{n109}
\ncangle[angleB=180,angleA=90,armB=0]{n106}{n108}
\ncangle[angleB=180,angleA=90,armB=0]{n104}{n106}
\ncangle[angleB=180,angleA=-90,armB=0]{n110}{n104}
\ncangle[angleB=180,angleA=-90,armB=0]{n112}{n111}
\ncangle[angleB=180,angleA=-90,armB=0]{n114}{n113}
\ncangle[angleB=180,angleA=90,armB=0]{n114}{n115}
\ncangle[angleB=180,angleA=90,armB=0]{n112}{n114}
\ncangle[angleB=180,angleA=-90,armB=0]{n116}{n112}
\ncangle[angleB=180,angleA=-90,armB=0]{n118}{n117}
\ncangle[angleB=180,angleA=-90,armB=0]{n120}{n119}
\ncangle[angleB=180,angleA=-90,armB=0]{n122}{n121}
\ncangle[angleB=180,angleA=-90,armB=0]{n124}{n123}
\ncangle[angleB=180,angleA=-90,armB=0]{n126}{n125}
\ncangle[angleB=180,angleA=-90,armB=0]{n128}{n127}
\ncangle[angleB=180,angleA=90,armB=0]{n128}{n129}
\ncangle[angleB=180,angleA=90,armB=0]{n126}{n128}
\ncangle[angleB=180,angleA=90,armB=0]{n124}{n126}
\ncangle[angleB=180,angleA=90,armB=0]{n122}{n124}
\ncangle[angleB=180,angleA=90,armB=0]{n120}{n122}
\ncangle[angleB=180,angleA=90,armB=0]{n118}{n120}
\ncangle[angleB=180,angleA=-90,armB=0]{n130}{n118}
\ncangle[angleB=180,angleA=-90,armB=0]{n132}{n131}
\ncangle[angleB=180,angleA=-90,armB=0]{n134}{n133}
\ncangle[angleB=180,angleA=90,armB=0]{n134}{n135}
\ncangle[angleB=180,angleA=90,armB=0]{n132}{n134}
\ncangle[angleB=180,angleA=90,armB=0]{n130}{n132}
\ncangle[angleB=180,angleA=90,armB=0]{n116}{n130}
\ncangle[angleB=180,angleA=-90,armB=0]{n136}{n116}
\ncangle[angleB=180,angleA=-90,armB=0]{n138}{n137}
\ncangle[angleB=180,angleA=-90,armB=0]{n140}{n139}
\ncangle[angleB=180,angleA=-90,armB=0]{n142}{n141}
\ncangle[angleB=180,angleA=-90,armB=0]{n144}{n143}
\ncangle[angleB=180,angleA=90,armB=0]{n144}{n145}
\ncangle[angleB=180,angleA=-90,armB=0]{n146}{n144}
\ncangle[angleB=180,angleA=-90,armB=0]{n148}{n147}
\ncangle[angleB=180,angleA=90,armB=0]{n148}{n149}
\ncangle[angleB=180,angleA=90,armB=0]{n146}{n148}
\ncangle[angleB=180,angleA=90,armB=0]{n142}{n146}
\ncangle[angleB=180,angleA=90,armB=0]{n140}{n142}
\ncangle[angleB=180,angleA=-90,armB=0]{n150}{n140}
\ncangle[angleB=180,angleA=-90,armB=0]{n152}{n151}
\ncangle[angleB=180,angleA=-90,armB=0]{n154}{n153}
\ncangle[angleB=180,angleA=90,armB=0]{n154}{n155}
\ncangle[angleB=180,angleA=90,armB=0]{n152}{n154}
\ncangle[angleB=180,angleA=90,armB=0]{n150}{n152}
\ncangle[angleB=180,angleA=90,armB=0]{n138}{n150}
\ncangle[angleB=180,angleA=90,armB=0]{n136}{n138}
\ncangle[angleB=180,angleA=90,armB=0]{n110}{n136}
\ncangle[angleB=180,angleA=-90,armB=0]{n156}{n110}
\ncangle[angleB=180,angleA=-90,armB=0]{n158}{n157}
\ncangle[angleB=180,angleA=-90,armB=0]{n160}{n159}
\ncangle[angleB=180,angleA=-90,armB=0]{n162}{n161}
\ncangle[angleB=180,angleA=90,armB=0]{n162}{n163}
\ncangle[angleB=180,angleA=90,armB=0]{n160}{n162}
\ncangle[angleB=180,angleA=-90,armB=0]{n164}{n160}
\ncangle[angleB=180,angleA=-90,armB=0]{n166}{n165}
\ncangle[angleB=180,angleA=-90,armB=0]{n168}{n167}
\ncangle[angleB=180,angleA=90,armB=0]{n168}{n169}
\ncangle[angleB=180,angleA=90,armB=0]{n166}{n168}
\ncangle[angleB=180,angleA=-90,armB=0]{n170}{n166}
\ncangle[angleB=180,angleA=-90,armB=0]{n172}{n171}
\ncangle[angleB=180,angleA=-90,armB=0]{n174}{n173}
\ncangle[angleB=180,angleA=90,armB=0]{n174}{n175}
\ncangle[angleB=180,angleA=-90,armB=0]{n176}{n174}
\ncangle[angleB=180,angleA=-90,armB=0]{n178}{n177}
\ncangle[angleB=180,angleA=90,armB=0]{n178}{n179}
\ncangle[angleB=180,angleA=-90,armB=0]{n180}{n178}
\ncangle[angleB=180,angleA=-90,armB=0]{n182}{n181}
\ncangle[angleB=180,angleA=-90,armB=0]{n184}{n183}
\ncangle[angleB=180,angleA=-90,armB=0]{n186}{n185}
\ncangle[angleB=180,angleA=-90,armB=0]{n188}{n187}
\ncangle[angleB=180,angleA=90,armB=0]{n188}{n189}
\ncangle[angleB=180,angleA=-90,armB=0]{n190}{n188}
\ncangle[angleB=180,angleA=-90,armB=0]{n192}{n191}
\ncangle[angleB=180,angleA=-90,armB=0]{n194}{n193}
\ncangle[angleB=180,angleA=90,armB=0]{n194}{n195}
\ncangle[angleB=180,angleA=90,armB=0]{n192}{n194}
\ncangle[angleB=180,angleA=90,armB=0]{n190}{n192}
\ncangle[angleB=180,angleA=90,armB=0]{n186}{n190}
\ncangle[angleB=180,angleA=90,armB=0]{n184}{n186}
\ncangle[angleB=180,angleA=90,armB=0]{n182}{n184}
\ncangle[angleB=180,angleA=90,armB=0]{n180}{n182}
\ncangle[angleB=180,angleA=-90,armB=0]{n196}{n180}
\ncangle[angleB=180,angleA=-90,armB=0]{n198}{n197}
\ncangle[angleB=180,angleA=-90,armB=0]{n200}{n199}
\ncangle[angleB=180,angleA=-90,armB=0]{n202}{n201}
\ncangle[angleB=180,angleA=90,armB=0]{n202}{n203}
\ncangle[angleB=180,angleA=90,armB=0]{n200}{n202}
\ncangle[angleB=180,angleA=90,armB=0]{n198}{n200}
\ncangle[angleB=180,angleA=-90,armB=0]{n204}{n198}
\ncangle[angleB=180,angleA=-90,armB=0]{n206}{n205}
\ncangle[angleB=180,angleA=-90,armB=0]{n208}{n207}
\ncangle[angleB=180,angleA=90,armB=0]{n208}{n209}
\ncangle[angleB=180,angleA=90,armB=0]{n206}{n208}
\ncangle[angleB=180,angleA=90,armB=0]{n204}{n206}
\ncangle[angleB=180,angleA=90,armB=0]{n196}{n204}
\ncangle[angleB=180,angleA=90,armB=0]{n176}{n196}
\ncangle[angleB=180,angleA=90,armB=0]{n172}{n176}
\ncangle[angleB=180,angleA=90,armB=0]{n170}{n172}
\ncangle[angleB=180,angleA=90,armB=0]{n164}{n170}
\ncangle[angleB=180,angleA=90,armB=0]{n158}{n164}
\ncangle[angleB=180,angleA=90,armB=0]{n156}{n158}
\ncangle[angleB=180,angleA=90,armB=0]{n98}{n156}
\ncangle[angleB=180,angleA=90,armB=0]{n64}{n98}
\ncangle[angleB=180,angleA=90,armB=0]{n32}{n64}
\ncangle[angleB=180,angleA=90,armB=0]{n14}{n32}
\ncangle[angleB=180,angleA=90,armB=0]{n2}{n14}
\end{pspicture}
}
      \end{center}
      \caption{Phylogeny of 105 strains of \emph{Eschericia coli}
	computed from whole genome sequences using
	\texttt{phylonium}~\cite{klo20:phy}. The scale bar is the number of
	substitutions per site. The clade marked ST131 contains the
	pathogenic targets ($\mathcal{T}$), the remaining seven
	strains are the neighbors ($\mathcal{N}$).}\label{fig:eco}
    \end{figure}


The program takes $k$ target sequences, $T=\{t_1,t_2,...,t_k\}$, and
$l$ neighbor sequences, $N=\{n_1,n_2,...,n_l\}$ as input. Then it
picks one target sequence, by default the shortest, as target
representative, $r$, and subtracts from $r$ all regions homologous to
the neighbors. The remaining target regions are a superset of the
desired markers.

It is convenient to think of sequences as sets of regions, so we can
write our set of putative markers, $m$, as the representative minus
the neighbors,
\begin{equation}\label{eq:fur1}
m \leftarrow r\setminus N.
\end{equation}
The marker set is then reduced by intersecting it with the targets
\begin{equation}\label{eq:fur2}
m \leftarrow m\cap T,
\end{equation}
and checked one more time for neighbor material in a second
subtraction step,
\begin{equation}\label{eq:fur3}
m \leftarrow m\setminus N.
\end{equation}
The first subtraction step is implemented using only exact
matching. It is thus fast, but perhaps not as sensitive as required
for the final output. This is generated in the second subtraction
step, which is implemented by comparing $m$ against $N$ using
Blast. The second subtraction is thus slower but more sensitive than
the first.

Steps (\ref{eq:fur1})--(\ref{eq:fur3}) summarize the \ty{fur}
algorithm. Among them, step~(\ref{eq:fur1}) is limiting as the exact
matching done here is based on the computation of an enhanced suffix
array~\cite{abo04:rep}. Naïvely, we could compute an enhanced suffix
array of $N$ and then stream $r$ against that. However, this would
require $O(|N|)$ memory with a large constant. For large $N$, this
quickly grows beyond the memory of conventional hardware, especially
if we are interested in analyzing mammalian genomes. However, the
sequences that make up $N$ can be analyzed sequentially making the
memory requirement proportional to the size of the longest neighbor
sequence, $O(\mbox{max}(|n_i|)$. Since we can always further subdivide
elements of $N$, \ty{fur} can be used to analyze very large data sets.

The program \ty{fur} doesn't operate directly on the target and
neighbor sequences. Instead, it reads a database constructed with the
program \ty{makeFurDb}, which is also part of this package. The
input to \ty{makeFurDb} is a directory of target sequences, and a
directory of neighbor sequences. Each sequence file in these
directories contains the genome of one taxon, perhaps in multiple
FASTA sequences representing contigs of chromosomes. The database
calculation from these sequences is a slow, one-off operation,
while \ty{fur} is normally quick and may be run repeatedly, with
various parameter combinations.

\begin{table}
\caption{The ten programs in the package \ty{fur}.}\label{tab:pro}
\begin{center}
\begin{tabular}{rll}
\hline
\# & Name & Purpose\\\hline
1 & \ty{checkPrim} & check primers through \emph{in silico} PCR\\
2 & \ty{cleanSeq} & remove runs of \ty{N} from \ty{fur} output\\
3 & \ty{fur2prim} & convert \ty{fur} output to \ty{primer3} input\\
4 & \ty{fur} & find unique regions\\
5 & \ty{madis} & match length distribution\\
6 & \ty{makeFurDb} & construct \ty{fur} database\\
7 & \ty{prim2fasta} & convert primers from \ty{primer3} output to FASTA\\
8 & \ty{prim2tab} & convert primers from \ty{primer3} output to table\\
9 & \ty{senSpec} & compute the sensitivity an specificity of \ty{fur}
output\\
10 & \ty{stream} & compare streaming strategies for \ty{makeFurDb}\\
\hline
\end{tabular}
\end{center}
\end{table}

Apart from \ty{makeFurDb} and \ty{fur}, this package also contains
six auxiliary programs for cleaning the output of \ty{fur}, assessing
its quality, and extracting primers from it (Table~\ref{tab:pro}). The
application of these auxiliary programs is also demonstrated in the
tutorial.
